word vs keywords 
chacha = words
for =  keywords 
for , for-each , loop , do-while, else , else-if,
let , var, const are the keywords 

-> what are keywords :-
Keywords are reserved words in a programming language that have special meaning and cannot be used as identifiers (such as variable names, function names, etc.). They are part of the syntax of the language and are used to perform specific operations. Examples of keywords in JavaScript include `for`, `while`, `if`, `else`, `let`, `const`, and `var`.

-> what is variable let and constants:-

Variables and constants are used to store data in a program. A variable is a named storage location that can hold a value, and its value can change during the execution of the program. In JavaScript, variables are declared using the `var`, `let`, or `const` keywords.

 where variables :- can be change during the execution of the program.
 where constants :- cannot be changed once they are assigned a value.
 where let :- 
 Block-scoped variables that can be updated but not re-declared 

Hosting:-
Hosting is a JavaScript mechanism that allows you to access variables and functions before they are declared in your code. This means that you can use a variable or function before it has been declared in your code. This is because JavaScript moves variable and function declarations to the top of their containing scope during the compilation phase, a behavior known as hoisting.

Undefined or not defined :-
Undefined :- is a special value in JavaScript that indicates that a variable has been declared but has not been assigned a value. It is a type in JavaScript that represents the absence of a value. On the other hand, "not defined" refers to a variable that has not been declared or is not in scope.
not defined :-
not defined refers to a variable that has not been declared or is not in scope. If you try to access a variable that has not been declared, you will get a ReferenceError.

Type in js :- 

Permitive and reference :-

Primitive types in JavaScript are:
1. Number (integer, float, double)
2. String
3. Boolean (true, false)
4. Undefined
5. Null
6. Symbol (new in ES6)

Reference types in JavaScript are:
1. Object
2. Function
3. Array
4. Date
5. RegExp
6. Error
7. Map
8. Set
9. WeakMap
10. WeakSet

Permitive = number,
Reference = object like () [] {} 

aisi koi bhi value jisko copy karne par real copy nahi hota, balki us main value kahte hai, aur jiska copy karne par real copy ho jaaye wo value perimitive tpye value ho hoti hai 

conditionals :- 
if, else if, else :- used to perform different actions based on the value of a condition.
switch :- used to perform different actions based on the value of a variable.
ternary operator :- used to assign a value to a variable based on a condition.

Loops :- 
for, while, do-while :- used to iterate over a collection of values.
for-of, for-in :- used to iterate over the values of an iterable object.
forEach :- used to iterate over the elements of an array.

Functions :-
function ka matlab aap kuch code ko likh kar koi naam de sakte ho iss liye function cames in action or or usey use kar sakte ho with the name as many as time you want 

functon = code ko name dena 

function mainly teen kaam ke liye hote hai
1. jab aapka code aap turant nahi chalana chaate future main chalaana chaate ho 
in this we use call back function 
2. jab aapka code aap reuse karana chate hoisting
3. jab aap code chalana chaate ho baar with different values or data 
that's why we use function.

params and arguments:-

arguments -: real value jo hum data hai function chalaate waqt 
params:- variable jisme values store hoti hai arguments waali

What are arrays:-

arrays = hum ek variable mein ek value store kar paata hai par jab humein ek se jaada value store karni ho to fir use hota hai array ka, matlab ki aaray aapko freedom deta hai ek sa jaada value use karne ki

arrays = group of values stored in a single variable

Now pop shift unshift:-

pop() -: remove last element from array and return that element
shift() -: remove first element from array and return that element
unshift() -: add element at the start of the array and return the new length of the array 

What is object:-
object = group of key value pairs stored in a single variable

